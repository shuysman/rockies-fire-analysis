\documentclass{article}
\usepackage[letterpaper,margin=1in]{geometry}
\usepackage{gensymb}
\usepackage{enumitem}
\usepackage{subcaption}
\usepackage{graphicx}
\graphicspath{
  {../img/}
}
\usepackage{hyperref}
\hypersetup{
  colorlinks=true,
  linkcolor=blue,
  citecolor=blue,
  urlcolor=blue
}
\usepackage[
style=apa,
natbib=true
]{biblatex}
\addbibresource{/home/steve/OneDrive/org/library.bib}
\author{Stephen Huysman}
\title{Wildfire Ignition Danger System for the Middle Rockies Ecoregion}

\begin{document}
\maketitle

% \tableofcontents

\section{Abstract}

\section{Introduction}

As a result of climate change, wildfire frequency and severity has increased across the western United States \citep{rileyRelationshipLargeFire2013}.  Climate projections show continued increases in frequency and severity into the future [citation needed].

Wildfire has important ecological consequences, such as altered forest structure, shifts in species composition, increased mortality, etc.

Fire management in national parks in a high priority and has consequences for the ecosystems managed by the parks as well as park visitors.

This analysis has several goals:

\begin{enumerate}[label=(\roman*)]
\item Reimplement the wildfire danger rating system for the Southern Rockies ecoregion developed by \citet{thomaWaterBalanceIndicator2020} using a different climate data source, gridMET, to facilitate the development of projections of fire risk.
\item Extend this wildfire danger rating system to a new location, the Middle Rockies Ecoregion.
\item Test whether the 14-day window of time used for rolling calculations in \citet{thomaWaterBalanceIndicator2020} is optimal or whether different lengths of time improve classification of fire risk.
\item Develop short-term forecasts and long-term projections of fire risk for the Southern and Middle Rockies Ecoregions.
\end{enumerate}
  
This work extends the wildfire ignition danger rating system implemented for the Southern Rockies level III ecoregion \citep{omernikEcoregionsConterminousUnited1987} from \citet{thomaWaterBalanceIndicator2020} to the Middle Rockies level III ecoregion.  

*Discuss advantages of ROC/threshold classifier approach.  Things I didn't really get from reading Thoma 2020*

\section{Methods}

Data on historical wildfires that occurred in the Middle Rockies Ecoregion from 1984 to 2020 were retrieved from the Monitoring Trends in Burn Severity (MTBS) database \citep{eidenshinkProjectMonitoringTrends2007}.  A total of 417 fires were included in the analysis, ranging from 1,003 to 563,527 acres in size.  For each fire polygon, a centroid was created using the ``Pole of Inaccessibility'' method in QGIS \citep{QGIS_software}, in order to ensure that for irregularly-shaped fire polygons the centroid point occurs somewhere within the polygon of burned area.  

Vegetation cover for each fire polygon was determined using the 2020 ``Existing Vegetation Type'' (EVT) data from the LANDFIRE database \citep{rollinsLANDFIRENationallyConsistent2009}.  For each each fire polygon in the MTBS dataset, the statistics tool in QGIS was used to determined mode EVT by pixel count.  The dominant vegetation type determined by this analysis was used to classify each fire polygon as occurring in ``forest'' or ``non-forest'' cover types.  

Using ignition dates provided by the MTBS data, wildfire seasons were defined for ``forest'' and ``non-forest'' cover types as the range of days of the year that fires historically have occurred on in those cover types within the ecoregion.

We compared climate and water balance variables representing energy and moisture as predictors of wildfire ignition.  Daily time series for historical climate variables representing measures of heat and moisture from 1979 through 2020 were retrieved from the gridMET gridded climate dataset \citep{abatzoglouDevelopmentGriddedSurface2013} for each centroid point determined for each historical wildfire (average temperature ($T$), average relative humidity ($RH$), and vapor pressure deficit ($VPD$)).  The complement of $RH$ (1 - $RH$) was taken to determine Relative Dryness $RD$, to allow direct comparison of magnitude with other dryness indicators. Historical water balance variables Actual Evapotranspiration ($AET$), Climatic Water Deficit ($D$), Soil Moisture ($SOIL$), and Rain ($RAIN$) were included from the National Park Service gridded water balance product \citep{tercekHistoricalChangesPlant2021}, which uses gridMET climate data as inputs.  $AWSSM$ calculated as $WHC$ - $SOIL$.  Growing Degree Days ($GDD$) were calculated from gridMET maximum and minimum daily temperature using a base temperature of 5.5\degree C.

We computed rolling sums or means for each of the climate and water balance variables, in order to represent accumulation of dryness over time leading to increased fire risk (Rolling sums: $RD$, $VPD$, $T$, $SOIL$, $AWSSM$; Rolling means: $RAIN$, $AET$, $D$, $GDD$).  Rolling calculations were preformed at windows of 1, 3, 5, 7, 9, 11, 14, 21, and 31 days to find the window of time that bests predicts fire risk.  Finally, the percentile rank of each $n$-day rolling calculation was taken to normalize variables to the local historical conditions for each pixel and to help account for variation in factors that were not modeled such as differences in stomatal resistance between vegetation types that would result in different responses to dryness.

Climate and water balance variable with the highest area under the curve (AUC) of the receiver operating characteristic (ROC) were determined to be the best overall classifiers of ignition.  

Once the best classifiers of ignition and the optimal window of time were determined, an ignition danger rating system was developed by modeling the proportion of historical wildfires that ignited at or below percentiles of the best performing climate/water balance variables. [eCDF]

We then used these historical distributions of water balance percentiles on day of ignition to develop near-term forecasts and long-term projections of fire ignition danger.  Near-term forecasts use cfsv2 forecast data from gridMET as inputs.  Long-term projections were developed for [n] GCMs provided by the MACA gridded climate data product \citep{abatzoglouComparisonStatisticalDownscaling2012}.  MACA data are downscaled using gridMET so the historical climate conditions developed using gridMET can compared directly to MACA conditions without bias correction.

\section{Results}

\subsection{Historical Fire Season \& Frequency in the Middle Rockies Ecoregion}

\begin{figure}[ht]
  \includegraphics[width=.75\textwidth]{Middle_Rockies/map-1.png}
  \caption{Historical wildfires larger than 1,000 acres occurring in the Middle Rockies Level III Ecoregion.  Centroid locations for MTBS fire polygons occurring on majority Forest and Non-Forest cover types are shown.}
  \label{fig:map}
\end{figure}

Of the 417 historical wildfires \textgreater1000 acres in size occurring in the Middle Rockies Ecoregion, 110 were determined to occur on ``forest'' and 307 on ``non-forest'' cover types (Figure \ref{fig:map}).  ``Forest'' wildfires occurred between 1985-07-12 and 2020-10-03.  ``Non-forest'' wildfires occurred between 1984-08-19 and 2020-10-30.  The largest ``forest'' wildfire was 563,527 acres and the largest ``non-forest'' wildfire was 303,427 acres.

The fire season for ``forest'' wildfires was determined to be between DOY 93 and 289.  The ``non-forest'' wildfire season was between DOY 6 and 350.  ucoFires were similarly distributed along the fire season in both forest and non-forest wildfires (Figure \ref{fig:fire-dens}). The mean day of year of ignition was 217.6 for forest and 215.4 for non-forest wildfires, with no evidence of difference in mean ignition day of year between the groups (Welch Two Sample t-test: t = 0.60794, df = 264.83, p-value = 0.5437).

\begin{figure}[ht]
  \includegraphics[scale=1]{Middle_Rockies/fire-distribution-1.png}
  \caption{Density plot of fire frequency by day of year, for forest and non-forest cover types.}
  \label{fig:fire-dens}
\end{figure}


\subsection{Rolling Window}
\subsection{Classifiers of Ignition}

For both Middle and Southern Rockies Ecoregions, measures of atmospheric dryness (CWD, VPD, and RD) were consistently the best overall classifiers of ignition as measured by overall AUC at all rolling window widths examined.  For forest cover types, the variables and rolling window widths that gave the best overall classification of fire ignition were CWD with 3 day windows for the Middle Rockies and RD with 17 day windows for the Southern Rockies Ecoregions.  For non-forest cover types, the best predictors were VPD with 1 day window (or daily) and RD with 17 day windows for the Middle and Southern Rockies Ecoregions, respectively.  However, choice of measure of atmospheric dryness and rolling window width appears to have minimal effect on overall ability to classify ignition, with RD, VPD, and CWD having similar AUC values for the rolling windows examined, with confidence intervals for AUC values overlapping.  As similar performance can be expected from any of these variables, modelers can be justified selecting one or another based on practical requirements such as data availability and ease of computation.

While measures of atmospheric dryness performed similarly to one another, they do appear to strongly outperform the other measures tested in classifying fire risk.   

\begin{figure}[ht]
  \centering
  \begin{subfigure}{.5\textwidth}
    \centering
    \includegraphics[width=\linewidth]{Southern Rockies-forest-3-days.png}
    \caption{Southern Rockies}
    \label{fig:sr-auc}
  \end{subfigure}%
  \begin{subfigure}{.5\textwidth}
    \centering
    \includegraphics[width=\linewidth]{Middle Rockies-forest-3-days.png}
    \caption{Middle Rockies}
    \label{fig:mr-auc}
  \end{subfigure}
  \caption{AUC Curves for Southern and Middle Rockies Ecoregions, 3 day rolling window.}
  \label{fig:auc}
\end{figure}

\section{Discussion}
\subsection{Historical Fire Season \& Frequency in the Middle Rockies Ecoregion}
\subsection{Classifiers of Ignition}
\subsection{Now-casts and near-term forecasts of fire ignition danger}
\subsection{Long-term projections of fire ignition danger}

\printbibliography[
heading=bibintoc,
title={Bibliography}
]
\end{document}

%%% Local Variables:
%%% mode: LaTeX
%%% TeX-master: t
%%% End:
