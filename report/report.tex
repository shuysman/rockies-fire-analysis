\documentclass{article}
\usepackage[letterpaper,margin=0.75in]{geometry}
\usepackage{hyperref}
\hypersetup{
    colorlinks=true,
    linkcolor=blue,
    citecolor=blue,
    urlcolor=blue
  }
\usepackage[
style=apa,
natbib=true
]{biblatex}
\addbibresource{/home/steve/OneDrive/org/library.bib}
\author{Stephen Huysman}
\title{Wildfire Ignition Danger System for Middle Rockies Ecoregion}

\begin{document}
\maketitle

\section{Abstract}

\section{Introduction}

As a result of climate change, wildfire frequency and severity has increased across the western United States \citep{rileyRelationshipLargeFire2013a}.  Climate projections show continued increases in frequency and severity into the future [citation needed].

Wildfire has important ecological consequences, such as altered forest structure, shifts in species composition, increased mortality, etc.

Fire management in national parks in a high priority and has consequences for the ecosystems managed by the parks as well as park visitors.


\section{Methods}

Data on historical wildfires that occurred in the Middle Rockies Ecoregion from 1984 to 2020 were retrieved from the Monitoring Trends in Burn Severity (MTBS) database \citep{eidenshinkProjectMonitoringTrends2007}.  A total of 417 fires were included in the analysis, ranging from 1,003 to 563,527 acres in size.  For each fire polygon, a centroid was created using the ``Pole of Inaccessibility'' method in QGIS \citep{QGIS_software}, in order to ensure that for irregularly-shaped fire polygons the centroid point occurs somewhere within the polygon of burned area.  

Vegetation cover for each fire polygon was determined using the 2020 ``Existing Vegetation Type'' (EVT) data from the LANDFIRE database \citep{rollinsLANDFIRENationallyConsistent2009}.  For each each fire polygon in the MTBS dataset, the statistics tool in QGIS was used to determined mode EVT by pixel count.  The dominant vegetation type determined by this analysis was used to classify each fire polygon as occurring in ``forest'' or ``non-forest'' cover types.  

We compared climate and water balance variables representing energy and moisture as predictors of wildfire ignition.  Daily time series for historical climate variables representing measures of heat and moisture from 1979 through 2020 were retrieved from the gridMET gridded climate dataset \citep{abatzoglouDevelopmentGriddedSurface2013} for each centroid point determined for each historical wildfire (precipitation ($P$), maximum daily temperature ($T_{max}$), minimum daily temperature ($T_{min}$), maximum relative humidity ($RH_{max}$), minimum relative humidity ($RH_{min}$), vapor pressure deficit ($VPD$)).  In addition, historical water balance variables Actual Evapotranspiration ($AET$) and Climatic Water Deficit ($D$) were included from the National Park Service gridded water balance product \citep{tercekHistoricalChangesPlant2021}, which uses gridMET climate data as inputs.

We computed rolling sums or means for each of the climate and water balance variables, in order to represent accumulation of dryness over time leading to increased fire risk (Rolling sums: $RD$, $VPD$, $T$, $SOIL$, $AWSSM$; Rolling means: $RAIN$, $AET$, $D$, $GDD$).  Rolling calculations were preformed at windows of 1, 3, 5, 7, 9, 11, 14, 21, and 31 days to find the window of time that bests predicts fire risk.  Finally, the percentile rank of each $n$-day rolling calculation was taken to normalize variables to the local historical conditions for each pixel and to help account for variation in factors that were not modeled such as differences in stomatal resistance between vegetation types that would result in different responses to dryness.

 highest area under the curve (AUC) of the receiver operating characteristic (ROC) 

\section{Results}

\section{Discussion}

\section{Bibliography}
\printbibliography

\end{document}
