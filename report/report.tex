\documentclass[11pt]{article}
\usepackage[letterpaper,margin=1in]{geometry}
\usepackage{gensymb}
\usepackage{enumitem}
\usepackage{subcaption}
\usepackage{xcolor}
\usepackage{graphicx}
\graphicspath{
  {../img/}
}
\usepackage{hyperref}
\hypersetup{
  colorlinks=true,
  linkcolor=blue,
  citecolor=blue,
  urlcolor=blue
}
\usepackage[
style=apa,
natbib=true
]{biblatex}
\addbibresource{/home/steve/OneDrive/org/library.bib}
\author{Stephen Huysman}
\title{Wildfire Ignition Danger System for the Middle Rockies Ecoregion}

\newcommand{\citethis}{\textbf{\textcolor{red}{[Citation Needed]}}}
\newcommand{\pauc}[1]{pAUC\textsubscript{#1}}

\begin{document}
\maketitle

% \tableofcontents

\begin{abstract}
A wildfire ignition danger rating system for the Middle Rockies ecoregion is presented. Rolling sums or means of climatic indicators of dryness were evaluated for their ability to classify wildfire ignitions at a range of rolling window widths from 1 to 31 days to identify the optimal width.  3 day rolling sums of Climatic Water Deficit (CWD) and Vapor Pressure Deficit (VPD) were the best performing classifiers of wildfire ignition. Wildfire ignition danger is determined by comparing the 3-day rolling sum of CWD to the distribution of historical rolling sums of CWD on days of wildfire ignitions in the Monitoring Trends in Burn Severity (MTBS) database, to compare observed conditions to conditions that were historically dry enough to burn in the ecoregion. Long-term projections of wildfire risk in the Middle Rockies made using this system show increased wildfire risk across the region by end-century (2070-2099) in both RCP4.5 and RCP8.5 emissions scenarios.
  
\end{abstract}

\section{Introduction}

The landscape of the western United States is shaped by wildfire, which alters the composition of plant communities, affects forage and shelter for wildlife, and can threaten human life and communities.  Wildfire severity and frequency has increased in recent years as a result of anthropogenic climate change \citep{abatzoglouImpactAnthropogenicClimate2016,runningGlobalWarmingCausing2006,boerChangingWeatherExtremes2017,littellReviewRelationshipsDrought2016}.  Increased frequency and severity of drought conditions drives these shifts, through changes such as earlier snow melt \citep{tercekForecasts21stCentury2016} and increased summer temperatures \citep{runningGlobalWarmingCausing2006}, although this view that drought and warming will universally result in more fire has been challenged \citep{littellClimateChangeFuture2018}.  In addition to these direct links to climate, wildfire occurrence and severity also depends on factors such as fuel load and ignition \citep{mckenzieClimateChangeEcohydrology2017}.

Understanding wildfire danger in public lands is a critical concern for land managers.  An understanding of fire risk both in the short and long term is necessary for managing plant and animal species and implementing fire management activities such as prescribed burns and fire restrictions. Short term wildfire risk impacts decision making such as allocation of resources for fire preparedness and response, prescribed burn windows, land use restrictions to minimize fire ignition risks, and evacuation plans for inhabited areas.  Over longer periods of time, wildfire can result in extensive changes to the landscape such as changes in plant community structures and even conversion to different forest types \citep{coopWildfireDrivenForestConversion2020}. Understanding long-term fire risk facilitates management activities such as reforestation with long-lived tree species, such as whitebark pine (\textit{Pinus albicaulis} Engelm.), which can take 50 years or longer to reach mature, cone-bearing age \citep{tombackWhitebarkPineCommunities2001}. Current whitebark pine planting efforts often target recently burned areas \citep{keaneRangewideRestorationStrategy2012}, with the implicit assumption that these areas will remain suitable into the future. Production of whitebark pine seedlings for restoration is an expensive and labor-intensive process, requiring identifying white pine blister rust resistant parent trees, collecting and growing out seeds in a nursery, and screening seedlings for rust resistance costing around \textbf{\$x} per seedling \citethis.  Optimal use of this limited resource requires selecting sites that will not burn again in the future before these seedlings can reach reproductive age. 

Wildfires occur when there is simultaneously fuel to to burn, conditions dry enough to burn, and an ignition source. A spectrum from fuel to moisture limited ecosystems has been proposed based on the climatic conditions determining fuel moisture and amount, with moisture-limited systems having ample fuel but with conditions often too wet to burn, and fuel-limited systems producing insufficient fuel to result in large fires even when dry conditions are present \citep{meynEnvironmentalDriversLarge2007}. Dry climatic conditions primarily affects wildfire ignition potential by creating conditions dry enough to burn, in addition to a longer-term role in shaping plant communities through drought stress that affects fuel loads \citep{littellReviewRelationshipsDrought2016}. Dry climatic conditions are correlated with burned area across the Western US, however the strength of a given measure of dryness as a predictor of burned area varies between vegetation types \citep{littellClimateWildfireArea2009}. Large wildfires are also promoted spatial continuity of available fuel, with conditions dry enough in different vegetation types across connected parts of the landscape \citep{millerConnectivityForestFuels2000}. Atmospheric dryness again plays a role in large wildfire formation here, by conditions dry enough to burn in moisture-limited vegetation types such as mesic forest which otherwise act as barriers to wildfire spread \citep{cawsonAtmosphericDrynessRemoves2024}.

\citet{thomaWaterBalanceIndicator2020} developed a wildfire ignition danger rating system for the Southern Rockies ecoregion \citep{omernikEcoregionsConterminousUnited1987} based on 14-day rolling sums of measures of atmospheric dryness: climatic water deficit (CWD) or vapor pressure deficit (VPD). Since both measures had similar performance as classifiers of ignition, \citet{thomaWaterBalanceIndicator2020} recommend the use of CWD over VPD because of its ease of calculation and availability in projected climate data.  Wildfire ignition danger is determined by comparing the rolling sum of CWD on a given day to the historical distribution of rolling sums of CWD on days that a wildfire ignition occurred. As such, it is a system that compares CWD on a given day to conditions that were actually dry enough to burn. The purely climate-driven nature of this ignition danger rating system facilitates the creation of long-term projections of wildfire ignition risk compared to other commonly used wildfire danger rating systems such as the National Fire Danger Rating System \citep{degrootChapter11Wildland2015}, which requires inputs that are either not available or difficult to simulate for future time periods such as wind speed and non-climatic inputs related to fuel load.

The aim of this study is to develop a wildfire ignition danger rating system for the Middle Rockies ecoregion that includes spatially explicit projections of ignition danger. Historical climate data are obtained from gridMET to facilitate the creation of long-term projections of fire risk using the MACA climate data source \citep{abatzoglouComparisonStatisticalDownscaling2012}.  The gridMET data were used to downscale MACA, allowing comparisons between the two data sources without bias correction \citep{tercekRobustProjectionsConsequences2023}.

% *Discuss advantages of ROC/threshold classifier approach.  Things I didn't really get from reading Thoma 2020*

% Need to add a figure showing relationship of dryness indicators + fire ignition.  For example CWD vs fire occurence, something causally linking dryness + fire.  Suggestion from David

\section{Methods}

Historical wildfire occurrence data in the Middle Rockies ecoregion \citep{omernikEcoregionsConterminousUnited1987} from 1984 through 2020 were retrieved from the Monitoring Trends in Burn Severity (MTBS) database \citep{eidenshinkProjectMonitoringTrends2007}.  A total of 417 fires were included in the analysis, ranging from 1,003 to 563,527 acres in size.  

Vegetation cover for each fire polygon was determined using the 2020 ``Existing Vegetation Type'' (EVT) data from the LANDFIRE database \citep{rollinsLANDFIRENationallyConsistent2009}.  For each fire polygon in the MTBS dataset, the statistics tool in QGIS was used to determined majority EVT by pixel count.  Each fire polygon was classified as ``forest'' cover type if the majority of pixels in the polygon were ``Tree'' type or ``non-forest'' cover type if the majority of pixels were ``Herb'', ``Sparse'', or ``Shrub''.

%% Using ignition dates provided by the MTBS data, wildfire seasons were defined for ``forest'' and ``non-forest'' cover types as the range of days of the year that fires historically have occurred on in those cover types within the ecoregion.

We compared climate and water balance variables representing energy and moisture as predictors of wildfire ignition. For each fire polygon, a centroid was determined using the ``Pole of Inaccessibility'' method in QGIS \citep{QGIS_software} in order to ensure that for irregularly-shaped fire polygons the centroid point is inside the burned polygon. Daily time series for historical climate variables representing measures of heat and moisture from 1979 through 2020 were retrieved from the gridMET gridded climate dataset \citep{abatzoglouDevelopmentGriddedSurface2013} for each centroid point determined for each historical wildfire (average temperature (T, \textit{\degree C}), average relative humidity (RH, \textit{\%}), and vapor pressure deficit (VPD, \textit{Pa})).  The complement of RH (100 - RH) was used to determine Relative Dryness RD, to allow direct comparison of magnitude with other dryness indicators. Historical water balance variables Actual Evapotranspiration (AET, mm), Climatic Water Deficit (D, \textit{mm}), Soil Moisture (SOIL, \textit{mm}), and Rain (RAIN, \textit{mm}) were obtained from the National Park Service gridded water balance product \citep{tercekHistoricalChangesPlant2021}, which uses gridMET climate data as inputs. Soil Water Holding Capacity (WHC, \textit{mm}) values were obtained from the US Natural Resources Conservation Service 
Soil Survey Geographic Database (SSURGO) product (\textbf{100 cm depth?  Not in Tercek et al 2021 or NPS Gridded WB model user manual pdf so might need to clarify with Mike}) \citep{naturalresourcesconservationserviceSoilSurveyGeographic2015}. Soil Water Deficit (SWD, \textit{mm}) was calculated as WHC - SOIL.  Growing Degree Days (GDD, \textit{\degree C}) were calculated from gridMET maximum and minimum daily temperature using a base temperature of 5.5\degree C \citep{mcmasterGrowingDegreedaysOne1997}.  

We computed rolling sums or means for each of the climate and water balance variables, in order to represent accumulation of dryness over time leading to increased fire risk.  Rolling means were calculated for variables which have an upper-bound: RD, VPD, T, SOIL, SWD. Rolling sums were computer for variables which have no upper-bound: RAIN, AET, D, GDD.  Rolling calculations were preformed at windows of 1, 3, 5, 7, 9, 11, 14, 21, and 31 days to identify the window of time that bests predicts fire risk.

Finally, the percentile rank of each n-day rolling sum or mean of the dryness indicators were taken to normalize variables to the local historical conditions for each pixel.  This transformation develops a relative indicator of dryness from the percentile rather than the absolute magnitude which is more difficult to estimate due to variation in factors that were not modeled such as differences in stomatal resistance between vegetation types that would cause varying rates of soil drying. 

A large proportion of the time series of rolling sums/means for several of the variables are zero, for example climatic water deficit is zero for many days during the winter.  % This creates a disjointed distribution of percentiles, where days with a rolling sum of zero have a percentile of zero, and any non-zero value for climatic water deficit could have a percentile of over 40, for example.  In order to reduce the disjointedness of the percentile rankings and prevent a situation where spatial projections of wildfire risk using our method have unrealistically sharp boundaries across pixels (leading to an appearance of ``speckling'' when the data are viewed spatially),
In order to minimize skewing of the distribution caused by zeros, we applied the following adjustments to the rolling sums/means before calculating percent ranks: 1) round all values to 1 decimal precision, 2) remove zeroes from the time series, and 3) apply a set operation to the values, so that there are no repeating values.  This last step is necessary to avoid inflating the percentile values. These adjustments % reduced the percentile inflation issue, as well as
removed the need for determining a fire season as in \citet{thomaWaterBalanceIndicator2020}.  Since days with a rolling sum of 0 mm of climatic water deficit would have no distinguishable fire risk by any interpretation of our model, our model considers any day with non-zero deficit (or other climatic variables) to be in the effective fire season instead of defining the fire season as a range of calendar dates (i.e., May 1st through October 31st).

Climate and water balance variables with the highest area under the curve (AUC) of the receiver operating characteristic (ROC) curve were determined to be the best overall classifiers of ignition (ignition occured on that date or not).  The ROC curve represents illustrates the performance of a binary classifier by plotting the trade-off between true- and false-positive rates at varying thresholds of classification \citep{pontiusRecommendationsUsingRelative2014}. Because misclassification (false negative) of fire risk under the driest conditions has the potential to be more costly than misclassifications under wetter conditions where fires are likely to be less severe, performance is also assessed using the partial AUC characteristic, \pauc{0.1} and \pauc{0.2}, which represent the area under the curve of the ROC for the range of false positive rates from 0 to 0.1 and 0 to 0.2, respectively. The pAUC values represent the classification performance under the driest conditions that resulted in large wildfire ignitions in the MTBS database.   

Once the best classifiers of ignition and the optimal window of time were determined, an ignition danger rating system was developed.  We estimated the proportion of historical wildfires that ignited at or below percentiles of the best performing climate/water balance variables by comparing observed values of n-day rolling sums/means of climate variables to the empirical cumulative distribution function (eCDF) for those values on days of observed wildfire ignitions. A third-degree polynomial curve was fit to the eCDF curves to generate a function to estimate the percentile of wildfires that burned at or below the input n-day rolling summed/averaged climate variable. 

We used this system to map wildfire ignition danger for forest cover out to the end of the century in the Middle Rockies ecoregion. Long-term projections of wildfire ignition risk were developed for 12 GCMs from the MACA gridded climate data product \citep{abatzoglouComparisonStatisticalDownscaling2012}. To characterize time spent at high levels of fire ignition danger, a threshold of dryness at which 10\% of historical fires in forest cover burned at below is used, .  Long-fire risk was characterized by averaging days above this threshold for each year in near-term (2023-2040), mid-term (2041-2060), and late-century (2081-2099) projection periods.  A threshold of 0.35 was selected for Figure \ref{fig:projected-risk}, which corresponds to the percentile of dryness (7 day rolling sum of D) at which 10\% of historical ignitions in forest cover types occurred or below.

\begin{table}[h!]
  \centering
  \begin{tabular}{ c }
    \hline
    GCM \\
    \hline
    BNU-ESM \\
    CanESM2 \\
    CCSM4 \\
    CNRM-CM5 \\
    CSIRO-Mk3-6-0 \\
    GFDL-ESM2G \\
    HadGEM2-CC365 \\
    inmcm4 \\
    IPSL-CM5A-LR \\
    MIROC5 \\
    MRI-CGCM3 \\
    NorESM1-M \\
    \hline
  \end{tabular}
  \caption{GCMs used to make projections of wildfire risk in the Middle Rockies ecoregion.  Ensemble projections of wildfire ignition danger were made using the mean days above a fire danger threshold for RCP4.5 and RCP8.5 emissions scenarios for all GCMs.}
  \label{table:gcms}
\end{table}

\subsection{Computing Environment}

Statistical analysis was performed using \texttt{R} version 4.4.2 \citep{rcoreteamLanguageEnvironmentStatistical2024} supported by the \texttt{tidyverse} packages \citep{hernangomezUsingTidyverseTerra2023} .  The R package \texttt{terra} was used for spatial data manipulation and analysis \citep{hijmansTerraSpatialData2024} and the \texttt{tidyterra} package was used to assist with plotting spatial data \citep{hernangomezUsingTidyverseTerra2023}.  Rolling means and sums were calculated using functions from the \texttt{zoo} package \citep{zeileisZooS3Infrastructure2005}.  

\section{Results}

\subsection{Historical Fire Season \& Frequency in the Middle Rockies Ecoregion}

\begin{figure}[ht]
  \includegraphics[width=.95\textwidth]{Middle_Rockies/map-1.png}
  \caption{Historical wildfires larger than 1,000 acres occurring in the Middle Rockies Level III Ecoregion. MTBS fire polygons occurring on majority Forest and Non-Forest cover types are shown.  Basemap attribution: © OpenStreetMap contributors © CARTO}
  \label{fig:map}
\end{figure}

Of the 417 historical wildfires greater than 4.047 km\textsuperscript{2}(1,000 acres) in size occurring in the Middle Rockies Ecoregion from 1984 through 2020, 110 occurred in forest and 307 in non-forest cover types (Figure \ref{fig:map}).  Forest wildfires occurred between 1985-07-12 and 2020-10-03.  Non-forest wildfires occurred between 1984-08-19 and 2020-10-30.  The largest forest wildfire was 563,527 acres and the largest non-forest wildfire was 303,427 acres.

Wildfires in forest-cover types occurred between DOY 93 and 289.  Non-forest wildfire wildfires occurred between DOY 6 and 350.  Fires were similarly distributed along the fire season in both forest and non-forest vegetation types (Figure \ref{fig:fire-dens}). The mean day of year of ignition was 217.6 for forest and 215.4 for non-forest wildfires, with no evidence of difference in mean ignition day of year between the groups (Welch Two Sample t-test: t = 0.60794, df = 264.83, p-value = 0.5437). \textit{I wrote this section before we decided to include all days in the model instead of only the wildfire season.  Not sure if it still fits in here.  At minimum I think I need to add discussion why we decided to change the definition of a fire season. -SJH}

\begin{figure}[ht]
  \includegraphics[scale=1]{Middle_Rockies/fire-distribution-1.png}
  \caption{Density plot of fire frequency by day of year, for forest and non-forest cover types.}
  \label{fig:fire-dens}
\end{figure}

\subsection{Climatic Classifiers of Wildfire Ignition}

Measures of atmospheric dryness (CWD, VPD, and RD) were consistently the best overall classifiers of ignition as measured by overall AUC at all rolling window widths examined compared to the other climatic measures examined (TEMP, SOIL, GDD, SWD, RAIN, and AET).  For forest cover types, the variables and rolling window widths that gave the best overall classification of fire ignition were CWD with 3 day windows.  For non-forest cover types, the best predictors were VPD with 1 day window (i.e., daily VPD) and RD with 17 day windows.  However, choice of measure of atmospheric dryness and rolling window width appears to have minimal effect on overall ability to classify ignition, with RD, VPD, and CWD having similar AUC values for the rolling windows examined, with confidence intervals for AUC values overlapping.  As similar performance can be expected from any of these variables, modelers can be justified selecting one or another based on practical requirements such as data availability and ease of computation.

\subsection{Rolling Window}
The different rolling window lengths tested had similar performance as measured by AUC and pAUC for most of the variables examined (Figure \ref{fig:auc_window}). AUC values for CWD and VPD were relatively constant across all rolling window lengths, however, pAUC values showed evidence of decreasing classification performance for these variables as rolling window length increases.  SOIL showed a decreasing pattern for all AUC and pAUC measures as rolling window length increased.  AET showed increasing classification performance as rolling window length increased.  


\begin{figure}[ht]
  \centering
  \begin{subfigure}{.5\textwidth}
    \centering
    \includegraphics[width=\linewidth]{Middle Rockies-non_forest-3-days.png}
    \caption{Non-forest}
    \label{fig:mr-nf-auc}
  \end{subfigure}%
  \begin{subfigure}{.5\textwidth}
    \centering
    \includegraphics[width=\linewidth]{Middle Rockies-forest-3-days.png}
    \caption{Forest}
    \label{fig:mr-f-auc}
  \end{subfigure}
  \caption{ROC Curves for forest and non-forest cover in the Middle Rockies Ecoregions, 3 day rolling window.}
  \label{fig:auc}
\end{figure}


\begin{figure}[ht]
  \centering
  \includegraphics[height=.75\textheight]{Middle_Rockies/auc_vs_window.png}
  \caption{AUC, \pauc{0.2}, and \pauc{0.1} values for different window widths used for calculations of rolling sums and means of variables shown for both forest and non-forest cover types. 95\% Confidence Intervals are shown for overall AUC values.  Confidence Intervals for pAUC values were not calculated due to computation constraints.}
  \label{fig:auc_window}
\end{figure}

\begin{figure}
  \centering
  \includegraphics[width=.95\textwidth]{Middle_Rockies/wildfire_risk_projections_gye.png}
  \caption{Ensemble projections of forest wildfire risk in the GYE (outlined with dotted blue line) using the model developed for Forest cover. The GYE is selected as a subset of the Middle Rockies Ecoregion to showcase wildfire risk projections using this method, as projections across the whole ecoregion were computationally intractable. Mean days above wildfire risk threshold (0.35 quantile of dryness) for RCP4.5 and RCP8.5 ensemble conditions shown. GCMs used to make ensemble average conditions are listed in Table \ref{table:gcms}. The 0.35 percentile of dryness (7 day rolling sum of deficit) corresponds to level of dryness at which approximately 10\% of fires burned at or below. Mean days per year above the fire risk threshold for periods 2021-2040, 2041-2060, and 2081-2099 are shown by the color ramp. Six points were assigned in a grid and values sampled for each interval shown, to show the changes in fire risk across time for the sampled locations.}
  \label{fig:projected-risk}
\end{figure}

\section{Discussion}

\subsection{Climatic Classifiers of Wildfire Ignition}
CWD and VPD were consistently the strongest classifiers of both forest and non-forest wildfire ignition for fires that burned at least 405 ha in the Middle Rockies ecoregion for almost all rolling window lengths examined (Figure \ref{fig:auc_window}), outperforming other climatic measures tested (AET, SWD, GDD, RAIN, RD, SOIL, and T).  These results are similar to the findings in \Citet{thomaWaterBalanceIndicator2020}, which identified CWD and VPD as the strongest classifiers of ignition using 7 day rolling sums or means in the Southern Rockies ecoregion.  In our analysis of the Middle Rockies, CWD and VPD had similar performance with overall AUC values around 0.8-0.9 for all rolling window widths between 1 and 31 days, with confidence intervals overlapping.  pAUC values for VPD for forest ignitions were relatively constant across the range of rolling window widths, while \pauc{0.1} values decreased as rolling window increased.  Given the consequences of false negatives when classifying fire risk are more severe under the driest conditions, this gives some support to using rolling window widths somewhere between 3-10 days if CWD would be used to assess wildfire ignition danger, in order to maximize \pauc{0.1} and classification performance under the driest conditions.

CWD is a measure of drought stress that is close to the dryness experienced by plants after accounting for simultaneous availability of water and energy needed for plant growth \citep{stephensonActualEvapotranspirationDeficit1998}. Vegetation composition influences wildfire behavior through the unique foliar physical and chemical properties of different species \citep{mattjollySeasonalVariationsRed2016}. Live fuels are not just wet dead fuels \citep{jollyPyroEcophysiologyShiftingParadigm2018}, and it seems that CWD and VPD are strong representations of dryness as related to the pyro-ecophysiology of the ecosystems studied here.

- \citet{littellClimateChangeFuture2018} challenged the dominance of a “drought and warming beget fire” approach to climate and fire in the western United States, arguing that simultaneous nonstationarity in climate, fire, and postfire fuel trajectories limits the predictability of fire area burned.
- \citet{littellClimateChangeFuture2018} - ecologically based climate-fire projections for western US - gradient from fuel-limited to flammability-limited fire regimes across ecosections.  Middle rockies would be
- \citet{zacharakisEnvironmentalForestFire2023} - VPD one of top performing indices globally

\subsection{Window width for rolling sums and means}

Our analysis found some evidence that shorter window widths for rolling sum and mean calculations resulted in better performance when classifying wildfire ignition, particularly under the driest conditions represented by the pAUC values. AUC values for CWD and VPD appeared constant across all rolling window widths, while pAUC values appeared to decrease with increasing window width (Figure \ref{fig:auc_window}). Other variables showed different patterns, such as RAIN which increased in overall AUC values for Forest cover types, but not for non-forest, when rolling window width was increased. However, these other climate variables did not result in optimal classification performance under any rolling window width when assessed with overall or partial AUC.  Therefore, to optimize performance classifying wildfire ignitions for locations in the Middle Rockies ecoregion, CWD should be used with a shorted rolling window width of around 3 days.  This results in similar performance to longer rolling window widths, such as the 14 day rolling window used for the Southern Rockies in  \citet{thomaWaterBalanceIndicator2020}, but results in better performance under driest conditions than longer windows for rolling calculations.

The effect of rolling window width on predictive performance appears to differ between forest and non-forest cover for some variables.  RAIN performs consistently across rolling window widths in non-forest, width AUC values around 0.5 and pAUC values around 0.0 for all rolling window widths (Figure \ref{fig:auc_window}) - the amount of time rainfall is allowed to accumulate in the model does not appear to either improve or decrease the model's ability to predict ignition. However, in forest cover, RAIN increases in predictive performance as rolling window increases and then levels off around 20 days (Figure \ref{fig:auc_window}). RAIN is overall a weaker predictor than other tested variables at all rolling window widths, but at longer rolling window widths approaches the performance of other variables in forest cover types. The different performance characteristics of RAIN in forest and non-forest cover types is possibly a result of differences between the vegetation present on the different cover types.  Since non-forest cover in the Middle Rockies would be dominated by shrubs and grasses with less woody mass, a larger proportion of the fuel biomass would be represented by fine herbaceous fuels instead of woody fuels \citethis. These finer fuels would less water, absorb water more quickly, and dry out more quickly \citep{vineyReviewFineFuel1991}, therefore, in forests this accumulation of rainfall is meaningful but in non-forest cover continued rainfall beyond the saturation point of the fuels has no effect.  RAIN appears to be particularly sensitive to these differences between forest and non-forest fuels compared to the other variables tested.

Some variables increase

% Increasing: AET, RAIN
% Decreasing: SOIL, D?
% Constant: VPD, T, RD, GDD

At the optimal rolling window width selected here, 3 days, overall performance of all classifiers tested besides RAIN as assessed by the AUC characteristic is higher for forest than for non-forest cover types (Figure \ref{fig:auc}). Performance under the driest conditions was also weaker for non-forest than forest cover, with lower pAUC values observed for the non-forest cover type (Figure \ref{fig:auc_window}, exact pAUC values are hard to read in this figure but they are lower for VPD and D at least, include table of all AUC/pAUC values for 3 day window?).  These results indicate that in the Middle Rockies, non-forest cover types are more fuel-limited and forest cover types are more moisture-limited in wildfire behavior \citep{meynEnvironmentalDriversLarge2007}. Wildfire behavior in moisture-limited ecosystems should be more sensitive to changes in atmospheric dryness because that is the force fundamentally limiting fire occurrence there since sufficient fuel is always present.  In grassland, shrubland, and other non-forest cover types, dryness alone is not sufficient to predict wildfire occurrence risk because the presence of sufficient fuel is not guaranteed.  However, performance of our classifiers in non-forest cover types was still strong, with D and VPD having AUC values of 0.92. Applications of this wildfire danger rating system should consider the inherently lower accuracy it will have in predicting occurrence in non-forested regions.  

\subsection{Long-term projections of wildfire ignition danger}

The ensemble averages (Table \ref{table:gcms}) of long-term projections of fire ignition danger in the Greater Yellowstone Ecosystem (GYE) within the Middle Rockies ecoregion made using this method show increased fire risk across the region by the end of the century under both RCP4.5 and RCP8.5 emissions scenarios, with larger increases in risk projected under the higher emissions RCP8.5 scenario (Figure \ref{fig:projected-risk}). The projected increases in fire risk across the GYE are not equal across the region.  For example, a forested location that could reach an average of 6 (RCP4.5) to 10 (RCP8.5) days above the fire risk threshold of 10\% by the near-term (2023-2040) could increase to an average of 23 (RCP4.5) to 61 (RCP8.5) days above the threshold by end-century (2081-2099), while another location could see larger absolute increases from 21-24 to 46-70 days for the same time periods and scenarios, respectively.  This potential increase in days of fire risk across most of the GYE presents a similar projection to \citet{westerlingContinuedWarmingCould2011}, who found large increases in burned area and decreases in fire return interval across the GYE.  

The selected risk threshold of conditions drier than the 3-day rolling sum of CWD at which 10\% of historical fires burned at or below represents a relatively low level of fire risk, resulting in conservative projections of fire risk.  This threshold can be tuned based on the risk tolerance required for decisions made using the model.  For example, an application requiring conservative estimates from the model would be the planting of slow-growing but high-value tree seedlings, such as whitebark pine, takes upwards of 50 years to reach cone-bearing age and therefore should be planted in locations that would not burn before the tree reaches maturity.

% \citet{mckenzieClimateChangeEcohydrology2017} challenges the view that hotter and drier conditions will universally result in increases in wildfire occurrence and severity. They found that the relationship between hot and dry conditions and wildfire is strong in mesic and arid forests and shrublands with substantial biomass, but weaker in the wetter or drier ends of a rainforest to desert gradient.  They concluded that regional drought-fire dynamics are not likely to be stationary in future climate and that accurate predictions of wildfire dynamics needs to consider vegetation changes  as well as changes in the drought-fire dynamic due to climate change. Our system is normalized across vegetation types within the ecoregion within forest and non-forest categories -> are our projections likely to suffer from this criticism?

The purely climate-based nature of this wildfire ignition danger rating system is an advantage that facilitates creating long-term projections of wildfire risk. Additionally, CWD can be calculated easily from readily available climate data or using pre-calculated datasets such as the NPS Gridded Water Balance product \citep{tercekHistoricalChangesPlant2021}, used here. The National Fire Danger Rating System \citep{degrootChapter11Wildland2015} which is commonly used by US land management agencies requires climate inputs such as wind speed and cloudiness, which are difficult to estimate in the long term, or non-climate inputs such as live fuel moisture, which cannot be known accurately in the long term, and therefore is limited to making short-term assessments of wildfire danger. Future work could compare the performance of this wildfire ignition danger rating system to the National Fire Danger Rating System. The performance of other climatic measures of drought used to estimate fire risk such as the Keetch-Byram Drought Index \citep{degrootChapter11Wildland2015} should also be compared to the performance of the classifiers assessed here.  This modeling should also be performed for other ecoregions to identify potential patterns or differences in fire dynamics between regions. 

\section{Acknowledgments}

Computational efforts were performed on the Tempest High Performance Computing System, operated and supported by University Information Technology Research Cyberinfrastructure at Montana State University.


\clearpage

\printbibliography[
heading=bibintoc,
title={References}
]
\end{document}

%%% Local Variables:
%%% mode: LaTeX
%%% TeX-master: t
%%% TeX-master: t
%%% TeX-master: t
%%% gptel-model: deepseek-r1:8b
%%% gptel--backend-name: "Ollama"
%%% gptel--bounds: ((8551 . 8552))
%%% TeX-master: t
%%% TeX-master: t
%%% TeX-master: t
%%% End:
