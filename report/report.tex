\documentclass[11p]{article}
\usepackage[letterpaper,margin=1in]{geometry}
\usepackage[none]{hyphenat} %% No hyphenation - to get cleaner docx conversion
\usepackage{gensymb}
\usepackage{tabularx}
\usepackage{adjustbox}
\usepackage{enumitem}
\usepackage[font=small,labelsep=period,labelfont=bf]{caption}
\usepackage{subcaption}
\usepackage{amsmath}
\usepackage{placeins}
\usepackage{afterpage}
\usepackage{float}
\floatstyle{plaintop}
\restylefloat{table}
\usepackage{xcolor}
\usepackage{graphicx}
\setkeys{Gin}{width=\linewidth}
\graphicspath{
  {../img/Middle_Rockies/}
}
\usepackage{hyperref}
\hypersetup{
  colorlinks=true,
  linkcolor=blue,
  citecolor=blue,
  urlcolor=blue
}
\usepackage[
style=authoryear-comp,
uniquename=false,
natbib=true
]{biblatex}
\addbibresource{/home/steve/OneDrive/org/library.bib}

\errorcontextlines=10

\author{Stephen Huysman}
\title{Wildfire Ignition Danger System for the Middle Rockies Ecoregion}

\newcommand{\citethis}{\textbf{\textcolor{red}{[Citation Needed]}}}
\newcommand{\pauc}[1]{pAUC\textsubscript{#1}}

\begin{document}
\maketitle

% \tableofcontents

\begin{abstract}
  A wildfire ignition danger rating system for the Middle Rockies
  ecoregion is presented. Rolling sums or means of climatic indicators
  of dryness were evaluated for their ability to classify wildfire
  ignitions at a range of rolling window widths from 1 to 31 days
  prior to ignition. Three-day day rolling sums of Climatic Water
  Deficit (CWD) and Vapor Pressure Deficit (VPD) were the best
  performing classifiers of wildfire ignition. Wildfire ignition
  danger was determined using the Monitoring Trends in Burn Severity
  (MTBS) database, by comparing the 3-day rolling sum of CWD to the
  distribution of historical rolling sums of CWD on days when fires
  started. This identified when conditions were historically dry
  enough to burn in forested and non-forested vegetation in the
  ecoregion. Long-term projections of wildfire risk in the Greater
  Yellowstone Ecosystem made using this system show increased wildfire
  risk across the region by end-century (2070-2099) in both RCP4.5 and
  RCP8.5 emissions scenarios.
  
\end{abstract}
% <2025-04-14 Mon> Call with David
% can we consider ecological benefit of fire? - yes, we just don't have that in our plans
% Fire management is like dominoes falling
% If one of the dominoes is fire would benefit the resource
% First domino to burn is dry enough to burn
% if time and interest: interesting question -
% - lambos are available, but we drive acura/prius because we dont need more and cant afford lambos
% - more expensive models soilwat, USGS charges 40% overhead, wont acknowledge project if less than 100k
% - project: compare time series of SFDI and do correlation with rolling deficit window 
% 3 day window is best: maybe fine fuels catch and start fires first

% Zack Holden papers - burned area with rain \citet{holdenDecreasingFireSeason2018} - RAIN related to burned area
% Jolly - fire intensity, ERC, area
% Riley

% rationale
% WB has already been demonstrated to work for:
% veg production, \citep{thomaWaterBalanceIndicator2020}, \citep{thomaSemiaridVegetationResponse2016},
% water need, \citep{thomaLandscapePivotPoints2019}
% amphibian occupancy in GRTE, wetland hydroperiod \citep{rayMultispeciesAmphibianMonitoring2022}, \citep{lafranceAmphibianRichnessRarity2024}
% streamflow (in prep, Thoma) \citep{thomaWaterBalanceIndicator2020}

% we've already shown useful in S Rockies for fire modeling, now we are expanding to new ecoregion with refinements

% simple approach advantages, cost effective, can be implemented anywhere with weather station or where gridded weather data exists. Can be implemented in a way that provides real-time automatic alerts
% couldn't find easy to access NFDS ratings for location of interest
% Allows manager to select level of risk tolerance. NFDS may be super accurate but is hard to interpret, not as easy to project. Ours forecast-able and project-able
% variables we're using for fire danger are also consistent with variables used for other biophysical processes- are easier to understand than measures like ERC, etc.

\section{Introduction}

The landscape of much of the western United States is shaped by wildfire, which alters the composition of plant communities, affects wildlife habitat, and can threaten human life and communities. Wildfire severity and frequency has increased in recent years as a result of anthropogenic climate change \citep{abatzoglouImpactAnthropogenicClimate2016,runningGlobalWarmingCausing2006,boerChangingWeatherExtremes2017,littellReviewRelationshipsDrought2016}. Increased frequency and severity of drought conditions drives these shifts, through changes such as earlier snow melt \citep{tercekForecasts21stCentury2016} and increased summer temperatures \citep{runningGlobalWarmingCausing2006}. In addition to these direct links to climate, wildfire occurrence and severity also depends on factors such as fuel load and ignition \citep{mckenzieClimateChangeEcohydrology2017}.

Understanding wildfire risk both in the short and long term is necessary for managing plant and animal species and implementing fire management activities such as prescribed burns and fire restrictions. Short term wildfire risk impacts decision making such as allocation of resources for fire preparedness and response, prescribed burn windows, land use restrictions to minimize fire ignition risks, and evacuation plans for inhabited areas. Over longer periods of time, wildfire can result in extensive changes to the landscape such as changes in plant community structure and under certain conditions result in conversion to different forest types \citep{coopWildfireDrivenForestConversion2020}. Understanding long-term fire risk facilitates management activities such as reforestation with long-lived tree species, such as whitebark pine (\textit{Pinus albicaulis} Engelm.), which can take 50 years or longer to reach mature, cone-bearing age \citep{tombackWhitebarkPineCommunities2001}. Current whitebark pine planting efforts often target recently burned areas \citep{keaneRangewideRestorationStrategy2012}, with the implicit assumption that these areas will remain suitable into the future. Production of whitebark pine seedlings for restoration is an expensive and labor-intensive process, requiring identifying white pine blister rust resistant parent trees, collecting and growing out seeds in a nursery, and screening seedlings for rust resistance \citep{tombackTammReviewCurrent2022}, costing approximately \$1,980 to \$2,400 USD per ha for this process including planting \citep{tombackMagnificentHighElevationFiveNeedle2011}. Optimal use of this limited and costly resource can be facilitated by selecting sites that are less likely to burn before these seedlings can reach reproductive age.

Wildfires occur when there is simultaneously dry fuel and an ignition source. A gradient from fuel to moisture limited ecosystems has been proposed based on the climatic conditions determining fuel accumulation and fuel moisture, with moisture-limited systems having ample fuel but with conditions often too wet to burn, and fuel-limited systems producing insufficient fuel to result in large fires even when dry conditions are present \citep{meynEnvironmentalDriversLarge2007}. Climate directly affects wildfire ignition potential by determining fuel moisture content, in addition to indirect long-term roles shaping plant communities and fuel loads \citep{littellReviewRelationshipsDrought2016}. Dry climatic conditions are correlated with burned area across the Western US, however the strength of a given metric of dryness as a predictor of burned area varies between vegetation types \citep{littellClimateWildfireArea2009}. Large wildfires develop when there is spatial continuity of available fuel, with conditions dry enough to burn % in different vegetation types across connected parts of the landscape
\citep{millerConnectivityForestFuels2000}. Atmospheric dryness plays a role in large wildfire formation, by creating conditions dry enough to burn in moisture-limited fire regimes, such as mesic forest which can otherwise act as barriers to wildfire spread \citep{cawsonAtmosphericDrynessRemoves2024}. 

Fire danger rating systems model the relationship between fuels, weather, and terrain to develop indices of wildfire potential such as probability of ignition, spread rates, and potential fire intensity. These systems inform decision making at local to international scales. The National Fire Danger Rating System (NFDRS) was first created in 1972 to standardize fire potential estimates across the U.S. and has supported operational decision making around wildfire danger for over 50 years \citep{jollyModernizingUSNational2024,zacharakisEnvironmentalForestFire2023}. The NFDRS incorporates modules for weather, fuel moisture models, and fuel models to estimate indices of fire danger ignition component (IC), spread component, burning index (BI), and energy release component (ERC) which are integrated to create a severe fire danger index that indicates potential for extreme fire events \citep{jollyModernizingUSNational2024,jollySevereFireDanger2019}. While the NFDRS produces a index of fire danger that is ultimately simple to interpret (commonly seen ``low'' to ``severe'' fire danger signs posted on US National Forest lands), it relies on complex inputs that require manual inputs and are sensitive to parameter calibration. Recent improvements to the NFDRS have removed the need for manual inputs of fuel dryness and attempt to make the system more compatible with gridded climate data instead of the point scale weather station data as it was originally designed to be used \citep{jollyModernizingUSNational2024}. Despite these improvements, implementation of spatial estimates of fire potential using widely available gridded climate data sets remains computationally intensive \citep{farguellFastSpatialNFDRS2025a}.

% IC operates as a proxy for the likelihood of a firebrand igniting fine fuels and becoming a fire that requires suppression
% Other metrics from the US National Fire Danger Rating System, including 10 and 100-hr dead fuel moisture and vapor pressure deficit, yielded results similar to those of IC,

The climatic water balance provides biologically relevant indicators of plant productivity and water availability that have demonstrated associations with plant distributions across spatial scales \citep{stephensonActualEvapotranspirationDeficit1998,stephensonClimaticControlVegetation1990}, plant productivity \citep{thomaSemiaridVegetationResponse2016,thomaWaterBalanceIndicator2020}, water need \citep{thomaLandscapePivotPoints2019}, amphibian occupancy and wetland hydroperiod \citep{rayMultispeciesAmphibianMonitoring2022,lafranceAmphibianRichnessRarity2024}, and streamflow \citep{thomaWaterBalanceIndicator2020}. Climatic water deficit (CWD), a measure of evaporative demand not met by available water, has also shown to be a strong indicator of wildfire ignitions in the Southern Rockies ecoregion \citep{omernikEcoregionsConterminousUnited1987}, where 14-day rolling sums of measures of atmospheric dryness (climatic water deficit (CWD) and vapor pressure deficit (VPD)) were used to develop a wildfire ignition danger rating system for the ecoregion \citep{thomaWaterBalanceIndicator2020} analogous to the IC from the NFDRS. Since both measures had similar performance as classifiers of ignition, \citet{thomaWaterBalanceIndicator2020} recommend the use of CWD over VPD because of its ease of calculation and availability in projected climate data sets. Wildfire ignition danger is determined by comparing the rolling sum of CWD on a given day to the historical distribution of rolling sums of CWD on days that a wildfire ignition occurred. As such, it is a system that compares CWD on a given day to conditions that were actually dry enough to burn. The purely climate-driven nature of this ignition danger rating system facilitates the creation of long-term projections of wildfire ignition risk compared to other commonly used wildfire danger rating systems such as the NFDRS, which requires inputs that are either not available or difficult to simulate for future time periods such as wind speed and non-climatic inputs related to fuel load. In addition, its use of water balance variables that are consistent with other biophysical processes facilitates easier interpretation than less biologically relevant indicators of fire danger such as the energy release component (ERC).

This research develops a wildfire ignition danger rating system for the Middle Rockies ecoregion based on \citet{thomaWaterBalanceIndicator2020} that includes spatially explicit projections of ignition danger that can be used in conservation planning, such as selection of sites that could serve as fire refugia for whitebark pine. We test the assumption made by \citet{thomaWaterBalanceIndicator2020} that 14-day rolling windows of indicators of dryness optimally predict wildfire ignitions and examine rolling window widths from 1 to 31 days. We also modify the percentile ranking algorithm used by \citet{thomaWaterBalanceIndicator2020} to improve sensitivity of the system to low levels of dryness and spatial continuity of mapped projections of wildfire ignition danger. Historical climate data were obtained from gridMET to facilitate the creation of long-term projections of fire risk using the Multivariate Adaptive Constructed Analogs (MACA) climate data source \citep{abatzoglouComparisonStatisticalDownscaling2012}. The gridMET data were used to downscale MACA, allowing comparisons between the two data sources without bias correction \citep{tercekRobustProjectionsConsequences2023}.

% *Discuss advantages of ROC/threshold classifier approach. Things I didn't really get from reading Thoma 2020*

\section{Methods}

\subsection{Climate, fire, and vegetation data}

Historical wildfire occurrence data from 1984 through 2020 were retrieved from the Monitoring Trends in Burn Severity (MTBS) database \citep{eidenshinkProjectMonitoringTrends2007} in the Middle Rockies ecoregion \citep{omernikEcoregionsConterminousUnited1987}. A total of 417 fires were included in the analysis, ranging in size from 405.9 to 228,051.3 hectares.

Vegetation cover for each fire polygon was determined using the 2020 ``Existing Vegetation Type'' (EVT) from the LANDFIRE database \citep{rollinsLANDFIRENationallyConsistent2009}. For each fire polygon in the MTBS dataset, the statistics tool in QGIS was used to determine majority EVT by pixel count. Each fire polygon was classified as ``forest'' cover type if the majority of pixels in the polygon were ``Tree'' (n = 110) type or ``non-forest'' (n = 307) cover type if the majority of pixels were ``Herb'', ``Sparse'', or ``Shrub''. This method assumes that the cover type before the wildfire is the same as the EVT in the 2020 LANDFIRE database. Thus, in the present analysis, cover type was assumed unchanged before and after fire and no wildfire-driven conversion of cover type occurred \citep{coopWildfireDrivenForestConversion2020}.

%% Using ignition dates provided by the MTBS data, wildfire seasons were defined for ``forest'' and ``non-forest'' cover types as the range of days of the year that fires historically have occurred on in those cover types within the ecoregion.

We competed climate and water balance variables representing energy and moisture as predictors of wildfire ignition. For each fire polygon, a centroid was determined using the ``Pole of Inaccessibility'' method in QGIS \citep{QGIS_software} in order to ensure that the centroid point was inside the burned polygon. Daily time series for historical climate variables representing measures of heat and moisture from 1979 through 2020 (Average temperature (T, \textit{\degree C}), average relative humidity (RH, \textit{\%}), and vapor pressure deficit (VPD, \textit{Pa})) were retrieved from the gridMET gridded climate dataset \citep{abatzoglouDevelopmentGriddedSurface2013} for each centroid point determined for each historical wildfire. The complement of RH (100 - RH) was used to determine Relative Dryness RD, to allow direct comparison of magnitude with other dryness indicators. Historical water balance variables Actual Evapotranspiration (AET, mm), Climatic Water Deficit (CWD, \textit{mm}), Soil Moisture (SOIL, \textit{mm}), and Rain (RAIN, \textit{mm}) were obtained from the National Park Service gridded water balance product \citep{tercekHistoricalChangesPlant2021}, which uses gridMET climate data as inputs. Soil Water Holding Capacity (WHC, \textit{mm}) for the top 100 cm of the soil profile were obtained from the US Natural Resources Conservation Service 
Soil Survey Geographic Database (SSURGO) product \citep{naturalresourcesconservationserviceSoilSurveyGeographic2015}. Soil Water Deficit (SWD, \textit{mm}) was calculated as WHC - SOIL. Growing Degree Days (GDD, \textit{\degree C}) were calculated from gridMET average daily temperature using a base temperature of 5.5 \degree C, a temperature below which little biological activity occurs \citep{mcmasterGrowingDegreedaysOne1997}. 

\subsection{Indicators of ignition}

We computed rolling sums or means (together ``rolling values'') for each of the climate and water balance variables to represent accumulation of dryness over time preceding ignition dates and competed these rolling values for performance predicting wildfire ignition. Variables were categorized as ``state'' variables if they describe the current condition of the system or ``flux'' variables if they describe movement of water, heat, or a substance through the system \citep{wangInterplaysStateFlux2019,dingmanPhysicalHydrology2015}. Rolling means were calculated for state variables: RD, VPD, T, SOIL, SWD. Rolling sums were computed for flux variables: RAIN, AET, CWD, GDD. Rolling calculations were preformed at windows of 1, 3, 5, 7, 9, 11, 14, 21, and 31 days that represent durations and magnitudes of drying that precede fires.

The percentile rank of each n-day rolling values of the dryness indicators were determined to normalize variables to the local historical conditions for each centroid pixel. This transformation develops a relative indicator of dryness from the percentile rather than the absolute magnitude which is more difficult to mode accurately due to variation in factors that were not modeled such as differences in stomatal resistance between vegetation types or due to incorrect model parametrization that used climate and soil properties data from published sources but with unknown accuracy at local scales. This is a similar transformation of climate or weather data to the percentile calculations of ERC and BI in the NFDRS, which captures the spatial and temporal context of the variables and places them on comparable scales \citep{jollySevereFireDanger2019}.

The time series for these rolling values contained large numbers of zeroes and near-zero values, for example climatic water deficit was zero for many days during the winter. This created a disjointed distribution of percentiles, where days with a rolling sum of zero have a percentile of zero, and any value for climatic water deficit meaningfully greater than zero would have a percentile that is much larger than zero, i.e. around 40, with no percentile values in between (Figure \ref{fig:pct-issue}). It also caused all fires to occur at or above the 90th percentile causing our model poor sensitivity to detect ignition risk based on deficit. In order to reduce the disjointedness of the percentile values and prevent a situation where spatial projections of wildfire risk using our method have unrealistically sharp boundaries across pixels (leading to an appearance of ``speckling'' when the data are viewed spatially), we applied the following adjustments in order to the rolling values before calculating percentiles:

\begin{enumerate}
\item Round all values to 1 decimal precision.
\item Remove zeroes from the time series.
\item Remove duplicate values so there are no repeating values.
\end{enumerate}
   
% This last step is necessary to avoid inflating the percentile values. 
These adjustments % reduced the percentile inflation issue, as well as
also removed the need for determining a fire season as in \citet{thomaWaterBalanceIndicator2020}. Since days with a rolling sum of 0 mm of climatic water deficit would have low to no fire risk, our model considers any day with non-zero deficit (or other climatic variables) to define the fire season instead of defining the fire season as a range of calendar dates (i.e., May 1st through October 31st).

\begin{figure}[htbp]
  \centering
  \includegraphics[width=0.75\textwidth]{pct-issue-1.png}
  \caption{Percentile of three-day rolling sum of deficit for \texttt{WY4444510406019850727} wildfire centroid pixel for 2005, showing original percentile calculation (\texttt{dplyr::percent\_rank()} function in R) and adjusted percentile calculation described in our methods. Without our adjustments, percentile values jump between 0 for a three-day rolling sum of deficit of zero and approximately 0.45 or greater for three-day rolling sum of deficit greater than zero. With adjustments, the percentile values are continuous from 0 to 100 and the model has higher sensitivity to lower levels of dryness, indicated by the increased range in percentile values observed over the year excluding the disjoint region in percentile values in the original percentile algorithm.}
  \label{fig:pct-issue}
\end{figure}

\subsection{Analysis}

We determined the best classifier of ignition as the variable and rolling window with the highest area under the curve (AUC) of the receiver operating characteristic (ROC) curve. The ROC curve represents the performance of a binary classifier (ignition or no ignition) by plotting the trade-off between true- and false-positive rates at varying thresholds of classification (Figure \ref{fig:roc}, \citet{pontiusRecommendationsUsingRelative2014}). Because misclassification (false negative) of fire risk under the driest conditions has the potential to be more costly than misclassifications under wetter conditions where fires are likely to be less severe, performance is also assessed using the partial AUC, \pauc{0.1} and \pauc{0.2}, which represent the area under the curve of the ROC for the range of false positive rates from 0 to 0.1 and 0 to 0.2, respectively. The pAUC values represent the classification performance under the driest conditions that resulted in large wildfire ignitions in the MTBS database.

\begin{figure}[htbp]
  \centering
  \includegraphics[width=.5\linewidth]{roc_curve.png}
  \caption{Receiver Operating Characteristic (ROC) curve showing true and false positive rates. Performance of a random classifier is shown by the diagonal line. Three example classifiers are showing in blue, green, and orange. The best possible classification performance is represented by the point in the upper left of the plot, which has 100\% sensitivity (no false negatives) and 100\% specificity (no false positives). \\ Image Source: cmglee, MartinThoma, CC BY-SA 4.0 $<$https://creativecommons.org/licenses/by-sa/4.0$>$, via Wikimedia Commons}
    \label{fig:roc}
\end{figure}

\subsection{Comparison with other indices of fire danger}

We compared our best performing classifier of ignition (three day rolling sum of CWD) to indices included in the NFDRS: 100-hr fuels (FM100), 1000-hr fuels (FM1000), burning index (BI), and energy release component (ERC).  FM100 and FM1000 are inputs to the IC of the FDRS, and should be most analogous to our model of wildfire ignitions \citep{jollyModernizingUSNational2024}. While the IC incorporates other measures besides fuel moisture, such as wind speed and the spread component, gridded values for IC that could be retrieved for a point of interest could not be located so the readily-available FM100 and FM1000 grids were used as a proxy for the IC. Daily time series from 1979-01-01 to 2021-12-31 for these indices calculated using gridMET were retrieved from Northwest Knowledge Network \citep{northwestknowledgenetworREACCHMETDATAGRIDMET2025} for the centroid polygon (Latitude 46.09374 
, Longitude -113.8503) for the Valley Complex (Coyote) fire (MTBS ID: \texttt{MT4609711384420000731}) which ignited on 2000-07-31. This fire was selected because its burned area (8336.9 ha) is approximately the mean burned area for forest cover in the Middle Rockies in the MTBS data base (8222.4 ha). Pairwise correlations were calculated for all variables using the \texttt{corrplot} function from the R GGally package \citep{schloerkeGGallyExtensionGgplot22024}.

\subsection{Wildfire danger rating}

Once the best classifiers of ignition and the optimal window of time were determined, an ignition danger rating system was developed. We estimated the proportion of historical wildfires that ignited at or below percentiles of the best performing climate/water balance variables by comparing observed values of n-day rolling values of climate variables to the empirical cumulative distribution function (eCDF) for those values on days of observed wildfire ignitions. A third-degree polynomial curve was fit to the eCDF curves to generate a function to estimate the percentile of wildfires that burned at or below the n-day rolling values of climate variable. This allowed us to convert a level of dryness to a statement that indicates the proportion of historic wildfires that burned at different levels of dryness. Conversion of the eCDF to a polynomial function also allows for portable implementation of the system elsewhere because it is a simple mathematical relationship relating percentile of dryness to historical proportion of wildfire ignitions. Figure \ref{fig:ecdf} illustrates the process of converting a proportion of historic wildfires to the level of dryness at which they burned at or below, which can be used as a threshold of dryness to determine risk of wildfire ignition.

\subsection{Projections of fire danger in the Greater Yellowstone Ecosystem}

We used this system to map wildfire ignition danger for forest cover for 20-year periods to the end of the century in the Greater Yellowstone Ecosytem (GYE), a subset of the Middle Rockies ecoregion. Projections are limited to the GYE and only made for forest cover types for the goal of identifying wildfire refugia for WBP in the GYE. Projections of wildfire ignition risk in forest cover types are made for all pixels in the region of interest, even those that may not currently have forest cover. This extrapolation of model predictions allows for projections of fire risk in areas where forest may be established through outplanting in places that are currently unforested.

Long-term projections of wildfire ignition risk were developed for 12 GCMs from the Multivariate Adaptive Constructed Analogs (MACA) gridded climate data product \citep{abatzoglouComparisonStatisticalDownscaling2012}. We used a threshold of 0.35 percentile of three-day rolling sum of deficit to characterize fire risk, the percentile of dryness at which 10\% of historical fires burned at or below (Figure \ref{fig:ecdf}). Days with projected three-day rolling sums of deficit above this threshold were considered at risk of ignition and number of days above the threshold for each year were summed. Long-term fire risk was characterized by averaging days above this threshold for each year in near-term (2023-2040), mid-term (2041-2060), and late-century (2081-2099) projection periods. 

\begin{table}[htbp]
  \centering
  \begin{tabular}{ c }
    \hline
    GCM \\
    \hline
    BNU-ESM \\
    CanESM2 \\
    CCSM4 \\
    CNRM-CM5 \\
    CSIRO-Mk3-6-0 \\
    GFDL-ESM2G \\
    HadGEM2-CC365 \\
    inmcm4 \\
    IPSL-CM5A-LR \\
    MIROC5 \\
    MRI-CGCM3 \\
    NorESM1-M \\
    \hline
  \end{tabular}
  \caption{GCMs used to make projections of wildfire risk in the Middle Rockies ecoregion. Ensemble projections of wildfire ignition danger were made using the mean days above a fire danger threshold for RCP4.5 and RCP8.5 emissions scenarios for all GCMs.}
  \label{table:gcms}
\end{table}

\subsection{Computing Environment}

Statistical analysis was performed using \texttt{R} version 4.4.2 \citep{rcoreteamLanguageEnvironmentStatistical2024} supported by the \texttt{tidyverse} packages \citep{hernangomezUsingTidyverseTerra2023} . The R package \texttt{terra} was used for spatial data manipulation and analysis \citep{hijmansTerraSpatialData2024} and the \texttt{tidyterra} package was used to assist with plotting spatial data \citep{hernangomezUsingTidyverseTerra2023}. Rolling values were calculated using functions from the \texttt{zoo} package \citep{zeileisZooS3Infrastructure2005}.  

\section{Results}

\subsection{Historical Fire Season \& Frequency in the Middle Rockies Ecoregion}

Out of the 417 historical wildfires greater than 405 hectares (1,000 acres) in size in the Middle Rockies Ecoregion from 1984 through 2020, 110 wildfires occurred in forest and 307 wildfires occurred in non-forest cover types (Figure \ref{fig:map}). Forest wildfires occurred between 1985-07-12 and 2020-10-03. Non-forest wildfires occurred between 1984-08-19 and 2020-10-30. The largest forest wildfire was 563,527 acres and the largest non-forest wildfire was 303,427 acres. The mean forest wildfire size was 8,222.4 ha and the mean non-forest wildfire size was 5,118.8 ha (Table \ref{tab:mtbs-stats})

\begin{figure}[htbp]
  \includegraphics[width=.95\textwidth]{map-1.png}
  \caption{MTBS wildfire polygons larger than 405 hectares occurring on forest and non-forest cover types in the Middle Rockies Level III Ecoregion (dark blue outline) between 1984 and 2020. Basemap attribution: © OpenStreetMap contributors © CARTO}
  \label{fig:map}
\end{figure}

\begin{table}
  \centering
  \begin{tabular}{lrrrrr}
    \hline
    Cover      & n   & Mean area (ha) & Min area (ha) & Max Area (ha) & SD (ha) \\
    \hline
    Forest     & 110 & 8222.4         & 416.0         & 228051.3      & 26807.1 \\
    Non-forest & 307 & 5118.8         & 405.9         & 122792.6      & 11860.8  \\
  \end{tabular}
  \caption{Statistics for wildfires classified as forest and non-forest in the MTBS database.}
  \label{tab:mtbs-stats}
\end{table}

Wildfires in forest-cover types occurred between day or year (DOY) 93 and 289. Non-forest wildfires occurred between DOY 6 and 350. Fire ignition DOYs had similar seasonal distributions in both forest and non-forest vegetation types (Figure \ref{fig:fire-dens}). The mean DOY of ignition was 217.6 for forest and 215.4 for non-forest wildfires, with no evidence of difference in mean ignition DOY between the groups (Welch Two Sample \textit{t} test: \textit{t}\textsubscript{264.83}=0.60794, \textit{P}=0.5437).

\begin{figure}[htbp]
  \includegraphics[scale=1]{dens-1.png}
  \caption{Density plot of fire frequency by day of year, for forest and non-forest cover types. Vertical dashed lines show mean day of ignition ($\mu_{\text{forest}} = 218$, $\mu_{\text{non-forest}} = 215$).}
  \label{fig:fire-dens}
\end{figure}

The classifiers of wildfire ignition tested here are distributed differently between days that experienced a wildfire ignition or not in the MTBS database (Figure \ref{fig:dens}). There is evidence that wildfires occurred more often on days with higher percentile values of the rolling values of dryness indicators (or lower values for SOIL, which is an indicator of moisture). However, the distribution of RAIN is dominated by low percentile values whether or not an ignition occurred.  

\begin{figure}[htbp]
  \centering
  \begin{subfigure}{.5\textwidth}
    \centering
    \includegraphics[width=\linewidth]{forest-dens-1.png}
    \caption{Non-forest}
    \label{fig:dens-nf}
  \end{subfigure}%
  \begin{subfigure}{.5\textwidth}
    \centering
    \includegraphics[width=\linewidth]{non_forest-dens-1.png}
    \caption{Forest}
    \label{fig:dens-f}
  \end{subfigure}
  \caption{Density plots showing distribution of percentiles for three-day rolling values on days with wildfire ignition (fire = 1), or no wildfire ignition (fire = 0) recorded in the MTBS database. Plot densities are scaled to 1 so that fire and non-fire distributions are on the same scale.}
  \label{fig:dens}
\end{figure}

\subsection{Climatic classifiers of wildfire ignition}

Measures of atmospheric dryness (CWD, VPD, and RD) were consistently the best overall classifiers of ignition as measured by overall AUC at all rolling window widths compared to the other climatic measures (TEMP, SOIL, GDD, SWD, RAIN, and AET). For forest cover types, the variables and rolling window widths that gave the best overall classification of fire ignition were CWD with a 3 day window (Figure \ref{fig:auc_window}). For non-forest cover types, the best predictors were VPD with a 1 day window (i.e., daily VPD) and RD with a 17 day window. However, among measures of atmospheric dryness (CWD, VPD, and RD), rolling window width appears to have minimal effect on overall ability to classify ignition, with RD, VPD, and CWD having similar AUC values for the rolling windows examined and overlapping AUC confidence intervals. As similar performance can be expected from these variables, selecting any one can be justified based on practical requirements such as data availability or ease of computation.

\subsection{Rolling window width}

The different rolling window widths tested had similar performance as measured by AUC and pAUC for most of the variables examined (Figure \ref{fig:auc_window}). AUC values for CWD and VPD were relatively constant across all rolling window widths, however, pAUC values showed evidence of decreasing classification performance for these variables as rolling window length increased. SOIL showed a decreasing pattern for all AUC and pAUC measures as rolling window length increased. AET showed increasing classification performance as rolling window length increased.  

\begin{figure}[htbp]
  \centering
  \begin{subfigure}{.5\textwidth}
    \centering
    \includegraphics[width=\linewidth]{Middle Rockies-non_forest-3-days.png}
    \caption{Non-forest}
    \label{fig:mr-nf-auc}
  \end{subfigure}%
  \begin{subfigure}{.5\textwidth}
    \centering
    \includegraphics[width=\linewidth]{Middle Rockies-forest-3-days.png}
    \caption{Forest}
    \label{fig:mr-f-auc}
  \end{subfigure}
  \caption{ROC Curves for a) non-forest and b) forest cover in the Middle Rockies Ecoregions, 3 day rolling window.}
  \label{fig:auc}
\end{figure}

\begin{figure}[htbp]
  \centering
  \includegraphics[scale=0.7]{auc-window-1.png}
  \caption{AUC, \pauc{0.2}, and \pauc{0.1} values for different window widths used for calculations of rolling sums and means of variables in forest or non-forest cover types. 95\% Confidence Intervals are shown for overall AUC values. Confidence Intervals for pAUC values were not calculated due to computational constraints because they require a lengthy bootstrapping process.}
  \label{fig:auc_window}
\end{figure}

\subsection{Comparison with other indices of fire danger}

The percentile of three day rolling sum of CWD for the Valley Complex (Coyote) fire showed strong positive correlation with ERC (corr = 0.8), weak positive correlation with BI (corr = 0.6), and strong negative correlations with FM100 (corr = -0.7) and FM 1000 (corr = -0.7) from 1979-01-01 to 2021-12-31 (Figure \ref{fig:coyote-corrplot}). The timing of ignition for the fire coincided with the peak maximum rolling sum of CWD and minimum FM100 for the year, and occurred slightly before the peak maximum BI, maximum ERC, and minimum FM1000 (Figure \ref{fig:coyote-fm}).

\begin{figure}[htbp]
  \centering
  \includegraphics[width = 0.5\textwidth]{coyote-corr-1.png}
  \caption{Pairwise Pearson correlations between percentile of three day rolling sum of CWD and NFDRS indices for the Valley Complex (Coyote) fire.}
  \label{fig:coyote-corrplot}
\end{figure}


\begin{figure}[htbp]
  \centering
  \includegraphics{coyote-fm-1.png}
  \caption{Daily time series of percentile of three day rolling sum of CWD and NFDRS indices for the Valley Complex (Coyote) wildfire. Red vertical line shows day of ignition (2000-07-31).}
  \label{fig:coyote-fm}
\end{figure}

\subsection{Wildfire ignition danger rating}

Wildfire ignition danger rating is determined by selecting a level of dryness that corresponds to a proportion of historical fires which burned at or below that level of dryness. We were interested in identifying wildfire refugia for whitebark pine using this model, so we identified a low level of fire risk to develop projections of wildfire danger. We selected the 35\textsuperscript{th} percentile of 3-day rolling sum of deficit, which corresponds to the level of dryness at which 10\% of historical fires on forest cover types burned at or below (Figure \ref{fig:ecdf}).

\begin{figure}[!htbp]
  \centering
  \includegraphics{ecdf-regression3-d-forest-1.png}
  \caption{Empirical cumulative distribution function (eCDF) three-day rolling sum of deficit for forest cover in the Middle Rockies ecoregion show in black with regression function estimating the cumulative distribution function (CDF) overlayed in red ($R^2 = 0.994$). The process to determine wildfire ignition danger is shown. A proportion of historical wildfire ignitions (y-axis) is selected based on risk tolerance for management objectives; 0.1 of historical wildfire ignitions is used here. The 0.35 percentile of three-day rolling sum of deficit corresponds to 0.1 of historical wildfire ignitions. When the percentile of a three-day rolling sum of deficit exceeds 0.35, it is considered at risk for wildfire ignition. Projections in Figure \ref{fig:projected-risk} average the sum of days above this threshold for each year in the periods projected.}
  \label{fig:ecdf}
\end{figure}

\subsection{Projections of wildfire ignition danger}

Ensemble averages (Table \ref{table:gcms}) of long-term projections of fire ignition danger in forests in the Greater Yellowstone Ecosystem (GYE) within the Middle Rockies ecoregion made using this method show increased fire risk across the region by the end of the century under both RCP4.5 and RCP8.5 emissions scenarios, with larger increases in risk projected under the higher emissions RCP8.5 scenario (Figure \ref{fig:projected-risk}). The projected increases in fire risk across the GYE are not equal across the region. For example, a forested location that could reach an average of 6 (RCP4.5) to 10 (RCP8.5) days above the fire risk threshold of 10\% by the near-term (2023-2040) could increase to an average of 23 (RCP4.5) to 61 (RCP8.5) days above the threshold by end-century (2081-2099), while another location could see larger absolute increases from 21-24 to 46-70 days for the same time periods and scenarios, respectively.

\begin{figure}[htbp]
  \centering
  \includegraphics[width=.95\textwidth]{wildfire_risk_projections_gye.png}
  \caption{Ensemble projections of forest wildfire risk in the GYE (outlined with dotted blue line) using the model developed for forest cover. The GYE is selected as a subset of the Middle Rockies Ecoregion to showcase wildfire risk projections using this method, as projections across the whole ecoregion were computationally intractable. Mean days above wildfire risk threshold (0.35 percentile of dryness) for RCP4.5 and RCP8.5 ensemble conditions shown. GCMs used to make ensemble average conditions are listed in Table \ref{table:gcms}. The 0.35 percentile of dryness (3 day rolling sum of deficit) corresponds to a level of dryness at or below which approximately 10\% of fires burned. Mean days per year above the fire risk threshold for periods 2021-2040, 2041-2060, and 2081-2099 are shown by the color ramp. Six points were assigned in a grid and values sampled for each interval shown, to show the changes in fire risk across time for the sampled locations.}
  \label{fig:projected-risk}
\end{figure}


\section{Discussion}

\subsection{Climatic Classifiers of Wildfire Ignition}
CWD and VPD were consistently the strongest classifiers of both forest and non-forest wildfire ignition for fires that burned at least 405 ha in the Middle Rockies ecoregion for almost all rolling window lengths examined (Figure \ref{fig:auc_window}), outperforming other climatic measures tested (AET, SWD, GDD, RAIN, RD, SOIL, and T). These results are similar to the findings in \Citet{thomaWaterBalanceIndicator2020}, which identified CWD and VPD as the strongest classifiers of ignition using 7 day rolling sums or means in the Southern Rockies ecoregion. In our analysis of the Middle Rockies, CWD and VPD had similar performance with overall AUC values between 0.8-0.9 for all rolling window widths between 1 and 31 days, with overlapping confidence intervals. The pAUC values for VPD for forest ignitions were relatively constant across the range of rolling window widths, while \pauc{0.1} values decreased as rolling window increased. Given the consequences of false negatives when classifying fire risk are more severe under the driest conditions, this gives some support to using rolling window widths somewhere between 3-10 days if CWD would be used to assess wildfire ignition danger, in order to maximize \pauc{0.1} and classification performance under the driest conditions.

CWD is a measure of drought stress that is close to the dryness experienced by plants as soil moisture declines and atmospheric demand continues, ultimately leading to soil moisture depletion \citep{stephensonActualEvapotranspirationDeficit1998}. Vegetation composition influences wildfire behavior through the unique foliar physical and chemical properties of different species \citep{mattjollySeasonalVariationsRed2016}. Live fuels are not just wet dead fuels \citep{jollyPyroEcophysiologyShiftingParadigm2018}, and it seems that CWD and VPD strongly represent dryness as related to the pyro-ecophysiology of the Middle Rockies ecoregion, especially in forest cover in this ecoregion.

The distribution of percentiles of three-day rolling values of dryness indicators showed differences between days when a wildfire ignition occurred and days without wildfire ignitions (Figure \ref{fig:dens}). The ability of CWD to discriminate days when ignition occurred is likely due to a strong association of days with wildfire ignition occurring when CWD percentile is high, while days with no ignition associated with low percentiles of deficit. It is interesting that VPD performs similarly to CWD in our analysis, because there is more overlap in distribution of VPD on days with and without ignition. However, unlike CWD, days with wildfire ignition almost always occur with a three day rolling mean of VPD above zero, while wildfires somewhat frequently occur with a three day rolling sum of CWD of zero. Therefore, percentiles VPD and CWD are likely defining different aspects of dryness as they relate to wildfire ignition potential and these differences are likely related to the different patterns of classifier performance measured by AUC or pAUC apparent as rolling window width varies (Figure \ref{fig:auc_window}).

- \citet{abatzoglouImpactAnthropogenicClimate2016} looked at PET, VPD, CWD, PDSI
- \citet{mckenzieClimateChangeEcohydrology2017} challenged the dominance of a “drought and warming beget fire” approach to climate and fire in the western United States, arguing that simultaneous nonstationarity in climate, fire, and postfire fuel trajectories limits the predictability of fire area burned.
- \citet{littellClimateChangeFuture2018} - ecologically based climate-fire projections for western US - gradient from fuel-limited to flammability-limited fire regimes across ecosections. Middle rockies would be
% - PET was the most frequent leading predictor (24%), followed by seasonal DEF (20%), TEMP (19%), PREC (17%), AET (14%), and SWE (4%)
% -Taken over the ecosections as a whole during the calibration period, a gradient of fire responses to deficit emerges, with mean area burned first increasing with mean warm season deficit, but then decreasing as deficit increases past about 375 mm
% -  Middle Rockies is high F-index in their work ( flammability limited)
% The most-limiting factors for area burned, even among relatively similar ecosections, are not stationary.
% Statistical projections of future climate-fire relationships based on historical dynamics are useful because they provide a relatively quick and simple approach to measure the relative sensitivity and potential for change in different ecosystems. They also likely have high skill in projecting shortterm changes in area burned (e.g., next 3–5 decades) before the altered climate-fire regime resets landscape-level fuel loads, and as such are useful in anticipating and adapting to likely large increases in fire area burned in much of the western United States. However, because future conditions eventually radically exceed the calibration domain, the models do not represent long-term forecasts.
% In ecosystems where fuels are chronically limiting, the usually slower rate of fuel accumulation under novel future climates is the limiting factor; fuels will still be flammable, perhaps for longer fire seasons, in most years. In ecosystems where flammability is chronically limiting, the climatic potential to render existing fuels available to burn is the limiting factor; fuels will still be present, and likely flammable more frequently and for longer fire seasons. Finally, it is plausible to anticipate that changes in human approaches to fire, including fire and vegetation management, land use, and population driven effects such as prevalence of ignitions could modify the controls we detected and push fire regime trajectories in directions we have not anticipated.
% \citet{heyerdahlClimateDriversRegionally2008} Climate in prior years was not a significant driver of regionally synchronous surface fires, likely because fuels were generally sufficient for the ignition and spread of such fires in these forests.
- \citet{zacharakisEnvironmentalForestFire2023} - VPD one of top performing indices globally
% \citet{littellClimaticWaterBalance2011} - Potential evapotranspiration, actual evapotranspiration, and deficit were usually better predictors in regression models of area burned than temperature or precipitation and their interactions.


% NFDRS
% disadvantages: sensitive to poorly calibrated parameters, leading to poor estimates of fire danger
% originally intended to be applied to point data from weather station data: not spatially explicit
% Extensive human input also precluded the system from leveraging gridded weather datasets to map and forecast fire danger where weather stations are absent, thus foregoing the opportunity to provide fire danger estimates at any point in the country > improved in version 4 \citep{jollyModernizingUSNational2024}
% The USNFDRS is composed of four primary modules: Inputs, Fuel Moisture Models, Fuel Models and Fire Danger Index Models

The purely climate-based nature of this wildfire ignition danger rating system is an advantage that facilitates creating spatial projections of wildfire risk. CWD can be calculated easily from readily available point data from weather stations, gridded climate data, or obtained from pre-calculated datasets such as the NPS Gridded Water Balance product \citep{tercekHistoricalChangesPlant2021} which was used here. The NFDRS \citep{degrootChapter11Wildland2015} which is commonly used by US land management agencies requires climate inputs such as wind speed and cloudiness, which are difficult to estimate in the long term, or non-climate inputs such as live fuel moisture, which cannot be known accurately in the long term, and therefore is limited to making short-term assessments of wildfire danger. The NFDRS is sensitive to poorly calibrated inputs, while our fire danger rating system is based off of a single climatic input and by its nature using percentile calculations is calibrated to the mechanisms of dryness relevant to ignition potential at each pixel. Recent improvements to the NFDRS have removed some need for manual inputs for the fuel moisture module \citep{jollyModernizingUSNational2024}, however, it remains computationally intensive to produce estimates of fire danger from gridded climate data \citep{farguellFastSpatialNFDRS2025a}. These improvements tot he NFDRS allow for spatial forecasting of the severe fire danger index component of the NFRDS \citep{jollySevereFireDanger2019}, however, its reliance on weather data such as wind speed and which are difficult to project accurately over longer terms may limit its use to short term forecasts and not long-term projections. Our fire danger rating system gives only an estimate of ignition potential, without equivalent estimates of other indices of fire danger provided by the NFDRS such as spread component, burning index, or energy release component. Therefore, we are able to estimate the potential for wildfire ignition, but not the potential for burned area or wildfire severity as in the NFRDS. However, our system needs only inputs of simple climate data making long-term spatial projections of wildfire risk readily calculable. Future work could compare the performance of this wildfire ignition danger rating system to the National Fire Danger Rating System. This modeling should also be performed for other ecoregions to identify potential patterns or differences in fire dynamics between regions. 


\subsection{Window width for rolling sums and means}

Our analysis found some evidence that shorter window widths for rolling sum and mean calculations resulted in better performance when classifying wildfire ignition, particularly under the driest conditions represented by the pAUC values. AUC values for CWD and VPD appeared constant across all rolling window widths, while pAUC values appeared to decrease with increasing window width (Figure \ref{fig:auc_window}). Other variables showed different patterns, such as RAIN which increased in overall AUC values for Forest cover types, but not for non-forest, when rolling window width was increased. However, these other climate variables did not result in optimal classification performance under any rolling window width when assessed with overall or partial AUC. Therefore, to optimize performance classifying wildfire ignitions for locations in the Middle Rockies ecoregion, CWD should be used with a shorted rolling window width of around 3 days. This results in similar performance to longer rolling window widths, such as the 14 day rolling window used for the Southern Rockies in  \citet{thomaWaterBalanceIndicator2020}, but results in better performance under driest conditions than longer windows for rolling calculations. (cite figure 4) (include wikipedia figure explaining roc)

The effect of rolling window width on predictive performance appears to differ between forest and non-forest cover for some variables. RAIN performs consistently poorly across rolling window widths in non-forest, width AUC values around 0.5 and pAUC values around 0.0 for all rolling window widths (Figure \ref{fig:auc_window}) - the amount of time rainfall is allowed to accumulate in the model does not appear to either improve or decrease the model's ability to predict ignition. However, in forest cover, RAIN increases in predictive performance as rolling window increases and then levels off around 20 days (Figure \ref{fig:auc_window}). RAIN is overall a weaker predictor than other tested variables at all rolling window widths, but at longer rolling window widths approaches the performance of other variables in forest cover types. The low performance of RAIN is supported by the lack of visible difference in the distribution of the percentile values of RAIN on days with or without wildfire ignitions (Figure \ref{fig:dens}). The different performance characteristics of RAIN in forest and non-forest cover types likely results from differences in vegetation fuel size. Since non-forest cover in the Middle Rockies is dominated by shrubs and grasses with less woody mass, a larger proportion of the fuel biomass is represented by fine herbaceous fuels instead of woody fuels. These finer fuels would store less water, absorb water more quickly, and dry out more quickly \citep{vineyReviewFineFuel1991}. Therefore, in forests this accumulation of rainfall is meaningful but in non-forest cover continued rainfall beyond the saturation point of the fuels has no effect. RAIN appears to be particularly sensitive to these differences between forest and non-forest fuels compared to the other variables tested.

Some variables increase in predictive performance as assessed by AUC as the rolling window width increases (AET, RAIN), some decrease in performance (SOIL), and some perform consistently (D, VPD, T, RD, GDD) (Figure \ref{fig:auc_window}). However, in some cases these patterns are not clear and at least for AUC values are within the overlap of 95\% confidence intervals, and some variables are constant in AUC but change in pAUC as rolling window changes. Of the variables that increase in performance, RAIN appears to level off and the relationship with fuel types was discussed above. AET, as a measure of the magnitude and length of growing conditions favorable to plants \citep{stephensonActualEvapotranspirationDeficit1998}, has poor predictive performance observed at the rolling window widths assessed here, and is likely more relevant to wildfire on longer time scales as a controlling factor on the productivity of vegetation; this is reflected in the continued increases in performance of AET as rolling window width increases in Figure \ref{fig:auc_window}. As a variable that decreases in AUC as rolling window width increases, SOIL is likely most relevant to wildfire ignition on shorter time scales, with longer rolling window widths smoothing out extremes in soil dryness that drive wildfire ignition. The variables that remain more or less constant in AUC are either measures of energy (T, GDD) or atmospheric dryness (D, VPD, RD). (I am having trouble explaining why these variables perform consistently across 1-31 days of time).

While rolling window width does not seem to affect the AUC of these variables, pAUC values do change with rolling width, and therefore are the more important characteristic to assess predictive performance.

% Increasing: AET, RAIN
% Decreasing: SOIL, D?
% Constant: VPD, T, RD, GDD

At the optimal rolling window width identified here (3 days) overall performance of all classifiers tested besides RAIN as assessed by the AUC characteristic is higher for forest than for non-forest cover types (Figure \ref{fig:auc}). Performance under the driest conditions was also weaker for non-forest than forest cover, with lower pAUC values observed for the non-forest cover type (Figure \ref{fig:auc_window}, exact pAUC values are hard to read in this figure but they are lower for VPD and D at least, include table of all AUC/pAUC values for 3 day window?). These results indicate that in the Middle Rockies, non-forest cover types are more fuel-limited and forest cover types are more moisture-limited in wildfire behavior \citep{meynEnvironmentalDriversLarge2007}. Wildfire behavior in moisture-limited ecosystems should be more sensitive to changes in atmospheric dryness because that is the force fundamentally limiting fire occurrence there since sufficient fuel is always present. In grassland, shrubland, and other non-forest cover types, dryness alone is not sufficient to predict wildfire occurrence risk because the presence of sufficient fuel is not guaranteed. However, performance of our classifiers in non-forest cover types was still strong, with D and VPD having AUC values of 0.92. Applications of this wildfire danger rating system should consider the inherently higher accuracy the system will have in predicting wildfire ignition danger in forested compared to non-forested regions.  

Consequences of misclassification of cover type at time of ignition. Fire driven cover change likely occurs in one direction, forest to non-forest due to aridification of the American West and the climate ratchet which controls the windows of time suitable for regeneration \citep{jacksonEcologyRatchetEvents2009}. The likelihood of misclassification puts fires that ignited in forests into non-forest categories which has the effect of the pulling down the likelihood of ignition in non-forests in the low deficit range and removing data points from forest ignitions, increasing uncertainty (refer readers to ecdf figure for non/forest fires). Unfortunately, this results in a fire danger rating in non-forest ratings under low-deficit conditions that is likely underestimating fire danger. Assuming every ignition we classified as non-forest was actually forest at time of ignition and experienced cover change, the maximum underestimation of fire danger would be 10\%, evident by visual inspection of the difference of proportion of historical wildfires occurring a 50th percentile of deficit or less between forest and non-forest cover types (ecdf figure reference). 

\subsection{Comparison with other indices of fire danger}

Ignition of the Valley Complex (Coyote) fire occurred around peak values for our top performing classifier of ignition, percentile of three day rolling sum of deficit, and selected indices from the NFDRS (FM10, FM100, BI, and ERC) are comparable (Figure \ref{fig:coyote-fm}). For this particular historical wildfire, peak three day rolling sum of deficit appears to most exactly correspond to day of ignition, while the peak values NFDRS indices besides FM100 are often slightly before or after the day of ignition. This indicates that FM100 most accurately captures the fuel moisture dynamics that were relevant to the Valley Complex fire, instead of the larger fuels driving FM1000. It is possible that in the Middle Rockies ecoregion in general, small fuels are the ignition source for fires even though larger, woody fuels are present.

The fact that the correlations between CWD and the NFDRS indices ranged from 0.6 to 0.8 (absolute value) indicates that our classifier captures different aspects of dryness as it relates to fire risk than these other indices. This is expected because CWD integrates simultaneous timing of energy and moisture as well as soil moisture reserves while the fuel moisture metrics model moisture availability in dead fuel, ERC models composite fuel moisture of dead and live fuels, and BI is an index of potential burn severity and effort needed to contain a single fire in a particular fuel type. The stronger correlations with BI and indicate that our percentiles of three day rolling sums of CWD should also perform well as an indicator of burned area and fire severity, however, these aspects of wildfire behavior were not examined here. 

The results of this comparison indicate that our simple model of fire ignitions compares favorably to the widely-used NFDRS. Our model based off of a simple water balance measure that can be calculated from readily available weather or climate data coincided with the exact timing of ignition in this example wildfire. Modeling fire risk with the NFDRS requires more complex inputs, which sometimes are manually input or themselves modeled values, and the individual indices tested here did not appear 

\subsection{Long-term projections of wildfire ignition danger}

Climate change is leading to longer and more intense fire seasons \citep{abatzoglouImpactAnthropogenicClimate2016,abatzoglouProjectedIncreasesWestern2021,littellReviewRelationshipsDrought2016,jollyClimateinducedVariationsGlobal2015}. This applies to the GYE specifically, where \citet{westerlingContinuedWarmingCould2011} found large increases in burned area and decreases in fire return interval across the ecoregion. Our projections of wildfire ignition risk in forests in the GYE support this claim, with projections showing increasing days at risk across the region by the end of the century, with larger increases in RCP8.5 than RCP4.5 ensemble projections.

The persistence of wildfire refugia that could be targeted for WBP planting depends on patterns of future emissions (Figure \ref{fig:projected-risk}). Mid-century projections show little difference in days at risk of wildfire ignition between the RCP4.5 and RCP8.5 ensembles, with only small increases over near-term projection wildfire risk apparent. The end-of-century RCP4.5 ensemble projections show likely persistence of low fire risk areas across high-elevations in the GYE in areas such as the Absaroka Range, Wind River Range, Crazy Mountains, and Tetons. However, increased CWD under the RCP8.5 ensemble projections shows virtually all areas in the GYE increasing in days at risk of wildfire ignition by the end of the century. Therefore, WBP plantings should target areas that show low fire risk under RCP4.5 projections, but the ultimate safety of these planted trees from wildfire is as uncertain as the pathway of future greenhouse gas emissions.

The selected risk threshold of conditions drier than the 3-day rolling sum of CWD at which 10\% of historical fires burned at or below represents a relatively low level of fire risk, resulting in conservative projections of fire risk. This threshold can be tuned based on the risk tolerance required for different management needs (consider managing for persistence vs managing for transition). For example, the application we focus on here which conservative estimates from the model is the planting of whitebark pine, a slow-growing species which takes 50 years or longer to reach cone-bearing age \citep{tombackWhitebarkPineCommunities2001} and therefore should be planted in locations that are unlikely to burn before the tree reaches maturity.

\citet{mckenzieClimateChangeEcohydrology2017} challenged the view that hotter and drier conditions will universally result in increases in wildfire occurrence and severity due to non-stationarity in the relationship between water, energy, and wildfire. They found that the relationship between hot and dry conditions and wildfire is strong in mesic and arid forests and shrublands with substantial biomass, but weaker in the wetter or drier ends of a rainforest to desert gradient. They concluded that regional drought-fire dynamics are not likely to be stationary in future climate and that accurate predictions of wildfire dynamics needs to consider vegetation changes  as well as changes in the drought-fire dynamic due to climate change. The accuracy of our projections of wildfire risk into the future therefore relies on the assumption that the forest and non-forest cover types present in the ecoregion will have the same relationship between the indicators of dryness and wildfire. However, our system is robust to spatial patterns of change of fuel abundance as long as sufficient fuel is available to burn as it doesn't depend on fuel quantity. Our system is normalized across vegetation types within forest and non-forest cover in the ecoregion, which means that projections of wildfire risk are likely to be robust as long as cover types shift within vegetation types currently observed within the ecoregion and that shifts to entirely new vegetation types are not observed.




\section{Acknowledgments}

Computational efforts were performed on the Tempest High Performance Computing System, operated and supported by University Information Technology Research Cyberinfrastructure (\texttt{RRID:SCR\_026229}) at Montana State University.

\clearpage

\printbibliography[
heading=bibintoc,
title={References}
]
\end{document}

%%% Local Variables:
%%% mode: LaTeX
%%% TeX-master: t
%%% TeX-master: t
%%% TeX-master: t
%%% End:
